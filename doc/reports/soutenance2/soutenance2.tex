\documentclass[a4paper,12pt]{article}
\usepackage[francais]{babel}
\usepackage[T1]{fontenc}
\usepackage[utf8]{inputenc}
\usepackage{pslatex}
\usepackage{url}
\usepackage{graphicx}
\usepackage{lscape}
\selectlanguage{francais}


\title{Rapport de Soutenance 2}
\author{
Ihy Group : \\
deguil\_x (Xavier Deguillard)\\
genite\_n (Nicolas Geniteau)\\
sezer\_s (Stephane Sezer)\\
wagnac\_t (Teddy Wagnac)
}

\begin{document}

\maketitle

\newpage

\section*{Introduction}

\newpage

\tableofcontents

\newpage

\section{Travail accompli}
	\subsection{Lecture du format}
Un pré requis la création d'un codec est sa lecture, le format doit
pouvoir être lu tel quel, et non pas décompressé vers un fichier wav, on
perdrait alors tous intérêt d'un tel format. Pour cela, plusieurs choses
ont été nécessaire. Premièrement, il fallait mettre en place un système
de ``streaming'', pour lire de façon fluide, en effet la décompression
doit se dérouler en même temps que la lecture. Une implantation
séquentielle naïve n'aurait pas eu l'effet escompte : s'il faut attendre
la fin de la lecture d'un ``chunk'' pour commencer a décompresser la
suite, pour finalement lire le chunk décompressé, l'utilisateur aurait
senti un temps d'arrêt lors de la lecture du fichier, en effet, la
décompression n'est pas instantanée.\\
La solution choisie ici est de faire la lecture et le décodage simultanément,
ainsi, on peut lire un chunk, et lorsque celui ci a fini de lire, on peut
directement enchainer sur le chunk suivant, puisque celui ci a été décompressé.
Pour mettre en place cette solution, il nous a fallu penser de façon parallèle,
en effet si la programmation séquentielle va de soit, il n'en est pas de même
pour la programmation parallèle.\\
Ce qui ici est essentielle, est une structure de données partagées entre les
deux processus (un qui décompresse, un autre qui lit). Pour cela, un buffer est
tout approprié. En effet, lorsque le processus de lecture a besoin de données
décompressées, il va simplement récupérer le prochain élément du buffer, et
lorsque le processus de décompression a fini de décompresser, il va tout
simplement ajouter les données dans le buffer. On voit ici clairement qu'un
buffer n'est qu'une simple structure de donnée de type ``fifo''\footnote{first
in first out}, qui est souvent implanté avec une file.\\
Notre buffer est néanmoins plus complexe qu'une simple file, en effet, il faut
qu'il puisse gérer les accès concurrentiels de la part des deux processus. Pour
cela, on utilise des ``mutex'' pour protéger les zones critiques, plus
précisément, l'ajout et le retrait d'un élément. De plus, pour des raisons
d'économies de mémoire, il est judicieux que le processus de décompression ne
soit pas toujours actif, en effet, lorsqu'un chunk est décompressé, il contient
uniquement des échantillons, tels qu'ils sont codés dans un fichier wav, tout
décompresser en une seule fois reviendrai a avoir le fichier wav intégralement
en mémoire. En plus de cela, la lecture audio de notre format ne doit pas
impacter les performances de l'ordinateur. Pour régler ces deux soucis, il
existe une solution plutôt simple, définir une taille maximum pour le buffer, et
lorsque celui-ci est plein, le processus de décompression va tout simplement
attendre que ce dernier se vide. La ``bonne taille'' a été déterminée et est une
taille de buffer de trois, c'est-à-dire que le buffer ne peut contenir que
trois chunks.\\
Lorsque tout ceci fut mis en place, il ne restait plus qu'à utiliser ce que nous
avions commencé à mettre en place à la soutenance dernière, en prévision de
cette lecture en streaming, la lecture d'un fichier wav. Ici, lorsque l'on
décompresse le ihy, on récupère des bouts de wav, qui correspondent aux chunks
du format. Il ne reste alors plus qu'à les envoyer à libao, pour que ce dernier
nous le lise.
	\subsection{Le type half}
	\subsection{Ondelettes}
	\subsection{Threading}
	\subsection{Huffman suite et fin}
À la dernière soutenance, nous pouvions utiliser huffman, pour compresser des
données, néanmoins, nous n'avions pas écrit le code pour la décompression, et
donc nous ne pouvions pas savoir si cela marchait réellement, nous savions juste
que cela devrait marcher théoriquement. Nous avons donc écrit le code pour
décompresser Huffman, et bien sûr, cela n'a pas fonctionné dès le début (ça
aurait été trop beau). En fait, la technique utilisée à la compression, pour
l'écriture était mauvaise, après réécriture de cette partie du code, tout
marchait.\\
Contrairement à notre première idée, Huffman n'est pas appliqué sur tout le
fichier, il n'y a pas un arbre unique pour tout le fichier. En fait, chaque
chunk possède son propre arbre de Huffman, premièrement, cela permet d'avoir un
arbre de Huffman qui est beaucoup plus précis, en contrepartie, l'arbre de
Huffman doit être écrit pour chaque chunk. Il fallait donc trouver le bon
compromis, afin d'avoir une compression optimale. À cause des ondelettes, la
taille d'un chunk doit être une puissance de deux, les tests pour l'optimalité
de Huffman ont donc été assez facile à réaliser. Sans avoir fait de tests (et
même avant d'avoir commencer Huffman), nous avions fixé la taille d'un chunk
comme étant équivalente à 65536 échantillons, et coup de chance pour nous, il
s'agit de la taille optimale pour Huffman, un chunk deux fois plus gros, ou deux
fois plus petit, grossis la taille du fichier final de quelques
kilo-octets.\\
	\subsection{Interface graphique}
	\subsection{Site web}
	\subsection{Portage sur iPhone}
Aujourd'hui, les codecs audio sont de plus en plus utilises sur des
ordinateur possedant une puissance plus que limite, j'ai nomme les iPods
et derives. Pour qu'un codec soit considere comme utilisable, il faut
qu'il puisse etre utilise sur ces plateformes, cela montre qu'il n'est
pas necessaire d'avoir un ordinateur tres puissant pour pouvoir lire le
format. C'est egalement une preuve d'un codec bien realise, celui ci
etant alors portable et donc theoriquement lisible sur tous les
appareils.\\
L'annee derniere apple a sorti un SDK\footnote{software developpement
kit} permettant de developper des applications pour l'iPhone, ceci en
utilisant les outils ainsi que les docs officielles de facon gratuite.
C'est l'occasion revee de montrer que notre codec est a la fois
portable et leger.\\
		\subsubsection{Reunir les outils}
Premiere etape donc, absolument indispensable a la realisation du
portage, la mise en place d'un environnement de cross-compilation,
c'est-a-dire, un compilateur pouvant generer du code pour iPhone, qui
tourne sur un ordinateur classique. Le SDK de apple en fournit un, mais
celui-ci est compatible uniquement avec MAC OSX Leopard, ce que nous ne
possedons pas. Heuresement, la communaute ``underground'' de l'iPhone
est assez importante, en cherchant bien, on a reussi a trouver un
tutoriel expliquant comment se construire un
``toolchain''\footnote{logiciel et headers complet pour compiler un
programme} integral. Apres cela, nous pouvions compiler et lancer un
simple Hello World sur l'iPhone.\\
Mais voila, notre projet n'est pas uniquement fait de C, mais egalement
de OCaml. Il a donc fallu trouver une facon de creer un compilateur caml
pour l'iPhone. Chose que l'on a egalement trouve sur internet, avec
tutoriel tres detaille.
		\subsubsection{Compiler le projet}
Notre projet ne possedant que tres peu de dependance externes, la
compilation s'est bien deroule, sauf sur la seule dependance : libao, la
bibliotheque nous permettant de lire du son. Apres l'avoir retire du
projet, nous avons teste le bon fonctionnement des autres composants du
projet, c'est-a-dire la compression et la decompression. Cela a
fonctionne du premier coup, chose a laquelle je ne m'attendais pas.\\

\newpage

\section{Tâches prévues}

\newpage

\section*{Conclusion}

\end{document}
