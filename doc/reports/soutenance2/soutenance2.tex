\documentclass[a4paper,12pt]{article}
\usepackage[francais]{babel}
\usepackage[T1]{fontenc}
\usepackage[utf8]{inputenc}
\usepackage{pslatex}
\usepackage{url}
\usepackage{graphicx}
\usepackage{lscape}
\selectlanguage{francais}


\title{Rapport de Soutenance 2}
\author{
Ihy Group : \\
deguil\_x (Xavier Deguillard)\\
genite\_n (Nicolas Geniteau)\\
sezer\_s (Stephane Sezer)\\
wagnac\_t (Teddy Wagnac)
}

\begin{document}

\maketitle

\newpage

\section*{Introduction}

\newpage

\tableofcontents

\newpage

\section{Travail accompli}
	\subsection{Lecture du format}
Un pre-requis a la creation d'un codec est sa lecture, le format doit
pouvoir etre lu tel quel, et non pas decompresse vers un fichier wav, on
perdrait alors tous interet d'un tel format. Pour cela, plusieurs choses
ont ete necessaire. Premierement, il fallait mettre en place un systeme
de ``streaming'', pour lire de facon fluide, en effet la decompression
doit se derouler en meme temps que la lecture. Une implentation
sequentielle naive n'aurait pas eu l'effet escompte : s'il faut attendre
la fin de la lecture d'un ``chunk'' pour commencer a decompresser la
suite, pour finalement lire le chunk decompresse, l'utilisateur aurait
senti un temps d'arret lors de la lecture du fichier, en effet, la
decompression n'est pas instantannee.\\
	\subsection{Le type half}
	\subsection{Ondelettes}
	\subsection{Threading}
	\subsection{Interface graphique}
	\subsection{Site web}
	\subsection{Portage sur iPhone}

\newpage

\section{Tâches prévues}

\newpage

\section*{Conclusion}

\end{document}
