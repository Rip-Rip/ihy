\documentclass[a4paper,12pt]{article}
\usepackage[francais]{babel}
\usepackage[T1]{fontenc}
\usepackage[utf8]{inputenc}
\usepackage{pslatex}
\usepackage{listings}
%\usepackage{times}

\title{Ihy format documentation}
\author{Ihy group}

\begin{document}

\maketitle

\begin{abstract}
Ici, sera présent les informations relatives au format Ihy, les headers, ainsi
que la façon dont le fichier est contruit, les différentes compression
utilisées dans un fichier audio au format ihy.
\end{abstract}

\tableofcontents

\section{Introduction}

Voici la structure du ihy, tel qu'il a été initialement implanté.

\lstset{numbers=left}
\begin{lstlisting}[frame=single]
typedef struct ihy_chunk
{
    uint32_t		ChunkSize;
    uint8_t		*Values;
} ihy_chunk;

typedef struct ihy_data
{
    /* Header */
    char		FileID[4]; /* "SNXT" */
    uint64_t		FileSize;
    uint16_t		CompressionType;
    uint8_t		Channels;
    uint32_t		Frequency;
    /* Tags */
    uint16_t		ArtistLength;
    char		*Artist;
    uint16_t		AlbumLength;
    char		*Album;
    uint16_t		TrackLength;
    char		*Track;
    uint16_t		Year;
    uint8_t		Genre;
    uint32_t		CommentLength;
    char		*Comment;
    /* Data */
    uint32_t		NbChunk;
    ihy_chunk		*DataChunks;
} ihy_data;
\end{lstlisting}

Les types de chaque champs renseignent la taille, ils ont été choisis pour être
invariant sur toutes les architecture.

\section{Headers}

Premièrement, le ``magic number'' du format, qui est la suite des 4 caractères
SNXT écrit en big-endian dans le fichier.
S'en suit le FileSize qui représente la taille totale en octet du fichier.
Le champ CompressionType sera détaillé plus bas, il donne les compressions
utilisées, à noter que plus le nombre de compressions augmente, plus la
décompression est lente, mais le fichier est alors de taille moindre.

\end{document}
