\documentclass[a4paper,12pt]{article}
\usepackage[francais]{babel}
\usepackage[T1]{fontenc}
\usepackage[utf8]{inputenc}
\usepackage{pslatex}
\usepackage{listings}
%\usepackage{times}

\title{Ihy format documentation}
\author{Ihy group}

\begin{document}

\maketitle

\begin{abstract}
Ici, sera présent les informations relatives au format Ihy, les headers, ainsi
que la façon dont le fichier est contruit, les différentes compression
utilisées dans un fichier audio au format ihy.
\end{abstract}

\tableofcontents

\section{Introduction}
Ce document a pour but de présenter le codec ihy, codec audio utilisant les
ondelettes. Il a pour principale cible les développeurs, qui pourront avec ce
manuel coder leur propre implémentation du codec.

\section{Qualité audio}
Le format ihy définit 4 niveaux de qualités, afin de répondre aux multiples
attentes de l'utilisateur final, selon son utilisation.\\
Le premier niveau correspond à un débit de 128kbits/s, sa qualité auditive est
très médiocre. Pour cette raison, nous ne conseillons pas d'utiliser cette
compression.\\
Le deuxième niveau de compression, qui équivaut à un débit de 192kbits/s, est de
bien meilleure qualité, on peut néanmoins ressentir avec de bons écouteurs, la
pauvre qualité du signal.\\
Le troisième correspond au débit binaire de 256kbits/s, qui est, dans
l'implémentation de référence, la qualité par défaut, est le bon compromis entre
qualité auditive et taille du fichier final.\\
Enfin le dernier niveau de compression, qui correspond à un débit de 320kbits/s
et de très bonne qualité et n'est recommandé qu'aux audiophiles.

\section{Format du header}
Un fichier ihy est composé avant tout d'un header principal sur tout le fichier,
ainsi qu'un header particulier pour chaque ``chunk''. Un fichier ihy étant
découpé en plusieurs ``chunks'' de taille fixe.\\
\subsection{Champs principaux}
Les 4 premiers octets du fichier, sont les octets de reconnaissance du fichier.
Il s'agit du ``magick number'' du ihy, qui est égal à : SNXT
Suit ensuite, 8 octets\footnote{Sauf contre indication, lorsque l'on parle d'une
taille, il s'agit d'un nombre non signé}
représentant la taille totale du fichier ihy, ce dernier
champs n'est pas indispensable et peut même valoir 0.\\
Le champs suivant qui est codé sur 2octets, correspond au type de compression
effectuée sur le fichier. Il s'agit avant tout d'un champ réservé pour de
futures applications et extensions du codec. Par défaut, lorsque ce champs est
nul, on y applique les compressions de base (voir plus bas).\\
Deux champs suivent enfin, le nombre de canaux, sur un octet ainsi que la
fréquence d'échantillonnage sur 4 octets.\\
\subsection{Tags audio}
Notre codec peut également contenir des tags audio, pour que l'identification
d'une piste soit facilitée. On a donc, dans l'ordre, sur 2 octets, la taille du
champ artiste qui le suit, sur 2octets encore, la taille du champs suivant
qui est le titre de la musique. Suivent l'année, sur 2 octets et le genre
musical sur 1
octet. Pour le genre, on se référera aux même genres que l'ID3. Enfin les deux
derniers champs sont dédiés aux commentaire, avec sur 4 octets la taille du
commentaire qui le suit.
\subsection{Les données}
Le header principal se finit sur 3 champs, surement les plus importants. Le
premier, sur 4 octets, représente la taille non compressée d'un chunk, un chunk
ayant une taille constante sur tout le fichier.\\
Le champs suivant, encore sur 4 octets, représente le nombre de chunk que le
fichier possède. Enfin, le dernier champ est un tableau de chunk et se finit
avec la fin du fichier.
\subsection{Format d'un chunk}
Commençons par le dernier champ, il s'agit des données compressées que l'on ne
peut pas réellement typer, sa taille est indiquée par le premier champ,
qui est
codé sur 4 octetss. Le champ suivant, qui est cod\'e sur 1 octet,
repr\'esente le nombre de bit de chaque coefficients quantifi\'e. Le
champ suivant, toujours indispensable \`a la quantification est le
facteur de quantification, il est cod\'e sur 2 octets. Enfin, l'avant
dernier champ repr\'esente la taille du dernier champ avant que l'on ne
le compresse avec Huffman, il est quant \`a lui cod\'e sur 4 octets.
\section{Algorithme utilis\'es}
\subsection{Les ondelettes}
Pour les ondelettes, il s'agit d'une simple ondelette de Haar, pour
laquelle, le coefficient du filtre miroir conjugu\'e, qui vaut
normalement $\frac{1}{\sqrt{2}}$, vaut ici $\frac{1}{2}$, ceci pour des
probl\'emes de pr\'ecisions vis-\`a-vis de la quantification.
\subsection{La quantification}
Il s'agit d'une simple quantification non lin\'eaire, qui a pour formule
:$$ k = sign\left(f\right) *
\left|\frac{f}{factor}\right|^{\frac{3}{4}}$$
et la décompression :
$$ f = sign\left(k\right) * \left|k\right|^{\frac{4}{3}} * factor$$
Avec $factor$ la puissance de la quantification.
\subsection{Huffman}
L'algorithme de Huffman est tr\`es simple, ce qui est particulier \`a
notre codec c'est la facon dont est \'ecrit l'arbre de Huffman.\\
Celle-ci est \'egalement tr\`es simple. Il s'agit d'effectuer un
parcours largeur de l\'arbre. Lorsque l'on rencontre un noeud, on
\'ecrit 0, lorsque l'on rencontre une feuille, on \'ecrit 1 puis le
caract\`ere correspondant \`a la feuille. Le tout occupe tr\`es
exactement 771. Apr\'es l'arbre de Huffman se trouve les caract\`eres
"Huffmani\'es".
\subsection{Ordre des algorithmes}
L'ordre des algorithmes va de soit. Il s'agit d'effectuer la
tranform\'ee par ondelette, puis de quantifi\'ees les coefficients selon
la qualit\'es voulue, puis de faire le passage par Huffman.

\section{Conclusion}
Vous avez desormais tous les \'elements en main pour r\'ealiser \`a la
fois un lecteur ihy, et un endodeur. Bon code.


\end{document}
