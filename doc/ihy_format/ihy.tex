\documentclass[a4paper,12pt]{article}
\usepackage[francais]{babel}
\usepackage[T1]{fontenc}
\usepackage[utf8]{inputenc}
\usepackage{pslatex}
\usepackage{listings}
%\usepackage{times}

\title{Ihy format documentation}
\author{Ihy group}

\begin{document}

\maketitle

\begin{abstract}
Ici, sera présent les informations relatives au format Ihy, les headers, ainsi
que la façon dont le fichier est contruit, les différentes compression
utilisées dans un fichier audio au format ihy.
\end{abstract}

\tableofcontents

\section{Introduction}
Ce document a pour but de présenter le codec ihy, codec audio utilisant les
ondelettes. Il a pour principale cible les développeurs, qui pourront avec ce
manuel coder leur propre implémentation du codec.

\section{Qualité audio}
Le format ihy définit 4 niveaux de qualités, afin de répondre aux multiples
attentes de l'utilisateur final, selon son utilisation.\\
Le premier niveau correspond à un débit de 128kbits/s, sa qualité auditive est
très médiocre. Pour cette raison, nous ne conseillons pas d'utiliser cette
compression.\\
Le deuxième niveau de compression, qui équivaut à un débit de 192kbits/s, est de
bien meilleure qualité, on peut néanmoins ressentir avec de bons écouteurs, la
pauvre qualité du signal.\\
Le troisième correspond au débit binaire de 256kbits/s, qui est, dans
l'implémentation de référence, la qualité par défaut, est le bon compromis entre
qualité auditive et taille du fichier final.\\
Enfin le dernier niveau de compression, qui correspond à un débit de 320kbits/s
et de très bonne qualité et n'est recommandé qu'aux audiophiles.

\section{Format du header}
Un fichier ihy est composé avant tout d'un header principal sur tout le fichier,
ainsi qu'un header particulier pour chaque ``chunk''. Un fichier ihy étant
découpé en plusieurs ``chunks'' de taille fixe.\\
Les 4 premiers octets du fichier, sont les octets de reconnaissance du fichier.
Il s'agit du ``magick number'' du ihy, qui est égal à :
\begin{lstlisting}[frame=single]
SNXT
\end{lstlisting}
Suit ensuite, 8 octets représentant la taille totale du fichier ihy.

\end{document}
